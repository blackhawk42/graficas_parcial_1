\section{Figuras geométricas}
\label{figuras_geometricas}

Todas las clases para figuras geométricas implementan al menos cuatro métodos:
\begin{itemize}
	\item Las funciones de transformación:
	\begin{itemize}
		\item \lstinline!void translation(int dx, int dy)!
		\item \lstinline!void scaling(double factor)!
		\item \lstinline!void rotation(double angle)!
	\end{itemize}
	\item La función de dibujado: \lstinline!void draw()!
\end{itemize}

Los detalles de implementación le corresponden a cada clase individual\footnote{Nótese
que esto es lo que en programación orientada a objetos se le llama una interfaz. Siguiendo
el consejo que se me dió en clase, decidí no luchar con C++ para implementarlo en el
código. Sin embargo, el hecho de que las estoy usando en todo pero en nombre demuestra
que mucho del poder del modelo orientado a objetos es simplemente en forzar al programador
a diseñar.}.

A pesar de esto, se sigue un patrón general en todas las figuras. Las funciones de
transformación simplemente llaman al \lstinline!PointTransformer! interno de la figura
y, úna vez calculada la matriz necesaria, la aplican a todos sus puntos. Todas las
figuras definen líneas, y el dibujo se logra simplemente llamando a la función
\lstinline!draw()! de cada una.

El pivote de las transformaciones es es definido al momento de creación y
fijo durante la vida del objeto. Esto siguiendo la recomendación vista en clase,
con el fin de mantenerlo simple. De ser necesario, sería trivial implementar métodos
para cambiar el pivote.

En todos los casos en los que se piden punteros a puntos, el punto es copiado a
una estructura interna; actualmente, instancias de \lstinline!Point! internas, aunque
podría cambiarse de ser necesario. Es decir, cambiar el punto externo del que se derivó
la figura no cambiará a la figura de ningún modo. Esto simplifica el manejo de memoria,
permite la reutilización de puntos externos y, mucho más importante, mantiene conceptos
de buen diseño como la encapsulación. De ser necesario, normalmente se implementan
métodos tipo \lstinline!get! para acceder a los puntos y cambiarlos directamente,
aunque obviamente se recomienda usar las funciones estándar cuando se pueda.

\subsection{La clase \lstinline!Circle!}

La clase \lstinline!Circle! implementa un círculo. Está definida en los archivos
\lstinline!Circle.h! y \lstinline!Circle.cpp!.

El constructor de la clase requiere tres parámetros: \lstinline!Point *origin!, el
punto de origen del círculo (i. e., el ``centro''), \lstinline!int radius!, el radio
del círculo, y \lstinline!resolution!, lo que he llamado la resolución.

Un círculo consiste en un número de puntos dibujados en torno al origen, usando
coordenadas polares. Cada punto se conecta al que le sigue por medio de una línea.
(El último punto se conecta al primero.) Una mayor resolución significa más puntos y,
por lo tanto, un círculo más suave.

Para dibujar el círculo, simplemente se itera la lista de lineas y, por cada una,
se le llama a su método \lstinline!draw()!.

Debido a su naturaleza dinámica, ésta es una de las pocas clases que requirió alojamiento
dinámico.

\subsection{La clase \lstinline!Triangle!}

La clase \lstinline!Triangle! implementa un triángulo. Está definida en los archivos
\lstinline!Triangle.h! y \lstinline!Triangle.cpp!.

Durante su creación, un triángulo sólo pide tres puntos arbitrarios, \lstinline!Point *p0!,
\lstinline!Point *p1! y \lstinline!Point *p2!. Esto se consideró más simple y flexible
que tratar de implementar toda clase de cálculos para definir triángulos arbitrarios.


\subsection{La clase \lstinline!Square!}

La clase \lstinline!Square! implementa un cuadrado. Está definida en los archivos
\lstinline!Square.h! y \lstinline!Square.cpp!.

Su constructor solicita el primer punto (la esquina inferior izquierda), \lstinline!Point p0!, y la longitud
de los lados del cuadrado, \lstinline!int length!.

Se decidió en contra de lados de diferentes longitudes simplemente para adherirse a la
especificación: se pidió un cuadrado, no rectángulos o cuadriláteros arbitrarios.
Por qué detenerse ahí: podrían implementarse polígonos arbitrarios con una lista dinámica
de puntos y... No, méjor pararle en lo que dicen las instrucciones, en algún lado
se tiene que hacer.
De ser necesario, no sería muy difícil extenderlo.