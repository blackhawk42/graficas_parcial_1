\section{Escenario}
\label{escenario}

Si se tuviera que definir una interfaz para diferentes tipos de escenarios, probablemente
consistiría simplemente en un método \lstinline!update()!, que actualiza todas las
figuras y variables de control internas y dibuja las figuras.

Actualmente existe sólo una clase en la categoría de ``escenarios'': la clase
\lstinline!Scene!.

\subsection{La clase \lstinline!Scene!}

La clase \lstinline!Scene! implementa un escenario. Está definida en los archivos
\lstinline!Scene.h! y \lstinline!Scene.cpp!.

El archivo \lstinline!main.cpp!, el punto de inicio del programa, inicia una sóla
instancia de ésta clase y llama a su método \lstinline!update()! en la función
\lstinline!draw_function()!, a su vez llamada por la función \lstinline!glutDisplayFunc()!,
cómo parte del ciclo de ejecución de OpenGL/GLUT.

El escenario consiste en 24 figuras geométricas y demuestra todos los tipos de transformaciones
en forma de simples animaciones.
Lamentablemente, nunca he tenido instinto artístico. Le aseguro que si no hice náda
más complejo o estético, fue por falta de tiempo y talento, no de ganas.
De referencia, las figuras son: una casa, un carro moviéndose, un árbol creciendo
y un hoja volando en el viento.

El escenario está definido de manera muy estática. La posición de las figuras y sus
animaciones son controladas de manera manual. Esto lo hice simplemente por simplicidad
y falta de tiempo, pero implementar clases para figuras específicas (\lstinline!Casa!, \lstinline!Arbol!,
etc.) sería sencillo de implementar.

Mi mayor lamentación fue que, debido a mi poca familiaridad con C++ (este es literalmente
mi primer proyecto en el lenguaje), no supe como inicializar arreglos en ciertas
circunstancias, sin hacerlos dinámicos. Debido a esto, el código no es tan limpio
como hubiera querido, pero traté de poner las instrucciones correctas en lugares lógicos.