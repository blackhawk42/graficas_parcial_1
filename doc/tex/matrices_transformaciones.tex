\section{Matrices y transformaciones geométricas}
\label{matrices_transformaciones}

\subsection{La clase \lstinline{Matrix}}

La clase \lstinline!Matrix! es la encargada de representar una matriz en general.
Está definida en los archivos \lstinline!Matrix.cpp! y \lstinline!Matrix.h!

Si bien sólo se requieren usar matrices de $3\times3$ y, $3\times1$ para este proyecto,
encontré que el diseño sería más simple de razonar si contaba con una matriz general.
Por lo tanto, esta clase puede, en teoría, representar una matriz válida de cualquier
dimensión, limitada sólo por la memoria del sistema. Dado que la matriz está en el
corazón de muchas de las operaciones importantes en el sistema, me pareció importante
que fuera tan robusta pero adaptable como fuera posible.

Los valores de la matriz son representados con tipos \lstinline!double!, lo que permite
trabajar con puntos decimales, especialmente en operaciones como escalado y rotación.
Estó, por supuesto, conlleva un costo en términos de eficiencia, pero en clase se nos
mencionó que éste costo era aceptable y necesario, al menos a nuestro nivel. Se encontró
que, en la práctica, esto no parece afectar demasiado al sistema.

Es posible que inadecuado manejo de errores de punto flotante sea la causa de un bug
menor en la rotación. Se hablará de esto más adelante.

La única operación matricial implementada es la de multiplicación. El método
\lstinline!Matrix *Matrix::multiply(Matrix *other)! toma un puntero a otra matriz,
\lstinline!other!, y la multiplica con la matriz que llamó al método\footnote{Cómo
se ve en la definición, el método está definido no para el objeto \lstinline!Matrix!,
sino para \emph{punteros} a un objeto \lstinline!Matrix!. Admito que esta es una práctica
que heredé de C y Go, que me ha funcionado y en la que razono bien, pero desconozco si
representa una buena o mala práctica en C++.}.

Se usa el algoritmo $O(n^3)$, el más simple y menos eficiente en términos de complejidad de tiempo.
Se nos informó en clase que éste sería suficiente, y en la práctica parece serlo.

Notablemente, esta clase es la única que requirió el uso de alocamiento dinámico de
memoria (i. e., las palabras clave \lstinline!new! y \lstinline!delete!). Esto
debido a la naturaleza dinámica de las matrices durante su creación y a que la multiplicación
regresa una matriz nueva, en vez de modificar otra como su resultado. Afortunadamente,
no le temo \emph{tanto} a los punteros (al menos en proyectos de este tamaño; un poco
más y correría a las colinas), y cuidados de manejo de memoria parecen haber evitado
fugas de memoria importantes.

Esta clase define únicamente la matriz general. Decidí que la lógica para interactuar
con ellas en términos gemétricos podría separarse en su propia clase, \lstinline!PointTransformer!.

\subsection{La clase \lstinline!PointTransformer!}

La clase \lstinline!PointTransformer! es la encargada de usar mátrices para realizar
transformaciones geométricas en puntos (véase \ref{primitivas_dibujo} para la explicación
de cómo se implementa un punto con la clase \lstinline!Point!).

La idea es que toda figura que desee realizar transformaciones en puntos puede incluir un
\lstinline!PointTransformer! como uno de sus atributos. Las clases para círculos,
triángulos y cuadrados incluyen uno (véase \ref{figuras_geometricas}). La API debe llamarse
en este orden:
\begin{enumerate}
	\item El pivote se asigna al momento de la construcción, con
	\lstinline!PointTransformer::PointTransformer(Point *pivot)!. (Por supuesto,
	existe también un \lstinline!Point *PointTransformer::getPivot()! y un \lstinline!void PointTransformer::setPivot(Point *p)!).
	\item La figura llama a la operación que desee realizar en sus puntos. Estas pueden
	ser:
	\begin{itemize}
		\item \lstinline!void PointTransformer::translation(int dx, int dy)!
		\item \lstinline!void PointTransformer::scaling(double sx, double sy)!
		\item \lstinline!void PointTransformer::rotation(double angle)!
	\end{itemize}
	\item Se llama a \lstinline!void PointTransformer::apply(Point *p)!, pasándosele
	el puntero al punto al que se le aplicará. Si hay varios puntos, se llama
	al método por cada uno.
\end{enumerate}

En un principio, se consideró una API capaz de encadenar muchas operaciones para ser
aplicadas de una vez. Decidí contra ello por limitaciones de tiempo y porque no se
consideró crucial para la eficiencia, pero sé por
seguro que refactorizar para hacerlo más flexible sería poco complicado.

Una importante optimización se da en las matrices de identidad. Cada \lstinline!PointTransformer!
contiene 4 matrices de identidad: la ``estándar'' (la verdadera
``matriz identidad'' de acuerdo a la definición matemática), y las identidades para
la traslación, rotación	y escalamiento vistas en clase. Me dí cuenta que, como la
API siempre se llama en el mismo orden, la transformación de una figura no interviene
con otras. Por lo tanto, las identidades pueden ser declaradas \lstinline!static!.
Si no estoy mal, esto significa que, en vez de 4 matrices por cada \lstinline!PointTransformer!
en el sistema (con 20 figuras, esto serían 80 matrices), las mismas 4 matrices
son compartidas entre todos los \lstinline!PointTransformer!, sólo alterando
los valores necesarios para cada transformación individual.

Este diseño reduce tremendamente el uso de memoria, que sería probablemente el
mayor inconveniente del proyecto. Por supuesto, tendría que hacerse una dura refactorización
si se necesitaran hacer varias transformaciones individuales al mismo tiempo (e. g.,
un sistema multihilo). Afortunadamente, no se mencionó nada sobre esto en los requerimientos,
por lo que se determinó que este sistema cumplía con su objetivo de eficiencia dadas
las circunstancias actuales.